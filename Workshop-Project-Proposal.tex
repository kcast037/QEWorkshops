% Options for packages loaded elsewhere
\PassOptionsToPackage{unicode}{hyperref}
\PassOptionsToPackage{hyphens}{url}
%
\documentclass[
]{article}
\usepackage{lmodern}
\usepackage{amssymb,amsmath}
\usepackage{ifxetex,ifluatex}
\ifnum 0\ifxetex 1\fi\ifluatex 1\fi=0 % if pdftex
  \usepackage[T1]{fontenc}
  \usepackage[utf8]{inputenc}
  \usepackage{textcomp} % provide euro and other symbols
\else % if luatex or xetex
  \usepackage{unicode-math}
  \defaultfontfeatures{Scale=MatchLowercase}
  \defaultfontfeatures[\rmfamily]{Ligatures=TeX,Scale=1}
\fi
% Use upquote if available, for straight quotes in verbatim environments
\IfFileExists{upquote.sty}{\usepackage{upquote}}{}
\IfFileExists{microtype.sty}{% use microtype if available
  \usepackage[]{microtype}
  \UseMicrotypeSet[protrusion]{basicmath} % disable protrusion for tt fonts
}{}
\makeatletter
\@ifundefined{KOMAClassName}{% if non-KOMA class
  \IfFileExists{parskip.sty}{%
    \usepackage{parskip}
  }{% else
    \setlength{\parindent}{0pt}
    \setlength{\parskip}{6pt plus 2pt minus 1pt}}
}{% if KOMA class
  \KOMAoptions{parskip=half}}
\makeatother
\usepackage{xcolor}
\IfFileExists{xurl.sty}{\usepackage{xurl}}{} % add URL line breaks if available
\IfFileExists{bookmark.sty}{\usepackage{bookmark}}{\usepackage{hyperref}}
\hypersetup{
  pdftitle={The Effects of C-100A on Tree Species and Water Quality within the Deering Estate High-flow and Low-flow Drainage Basins},
  pdfauthor={Katherine Castrillon},
  hidelinks,
  pdfcreator={LaTeX via pandoc}}
\urlstyle{same} % disable monospaced font for URLs
\usepackage[margin=1in]{geometry}
\usepackage{graphicx,grffile}
\makeatletter
\def\maxwidth{\ifdim\Gin@nat@width>\linewidth\linewidth\else\Gin@nat@width\fi}
\def\maxheight{\ifdim\Gin@nat@height>\textheight\textheight\else\Gin@nat@height\fi}
\makeatother
% Scale images if necessary, so that they will not overflow the page
% margins by default, and it is still possible to overwrite the defaults
% using explicit options in \includegraphics[width, height, ...]{}
\setkeys{Gin}{width=\maxwidth,height=\maxheight,keepaspectratio}
% Set default figure placement to htbp
\makeatletter
\def\fps@figure{htbp}
\makeatother
\setlength{\emergencystretch}{3em} % prevent overfull lines
\providecommand{\tightlist}{%
  \setlength{\itemsep}{0pt}\setlength{\parskip}{0pt}}
\setcounter{secnumdepth}{-\maxdimen} % remove section numbering

\title{The Effects of C-100A on Tree Species and Water Quality within the
Deering Estate High-flow and Low-flow Drainage Basins}
\author{Katherine Castrillon}
\date{01/10/2020}

\begin{document}
\maketitle

\begin{center}\rule{0.5\linewidth}{\linethickness}\end{center}

\hypertarget{research-statement}{%
\section{Research Statement}\label{research-statement}}

I am a Masters student under CREST with Dr.~Michael Ross as my advisor
within the Earth and Environment Department at Florida International
University. My research topic is focused within the boundary of the
Deering Estate at Palmetto Bay in Miami, FL. Utilizing HOBOware, I will
be monitoring the hydrology within the Cutler Slough Rehydration
Project. I am looking into understanding the effects of the CSRP's
unsystematic hydrologic regime and water quality on tree species present
within the Deering Estates Natural Areas wet limestone hammock
ecosystem.

\begin{center}\rule{0.5\linewidth}{\linethickness}\end{center}

\hypertarget{objectives-and-hypothesis}{%
\section{Objectives and Hypothesis}\label{objectives-and-hypothesis}}

The Deering Estate natural areas is composed of transitions in habitat
type from wet limestone hardwood hammock to red mangrove fringe forest.
The current restoration taking place, known as the Cutler Slough
Rehydration Project, was created to mimic the historical wetland's
hydrological regime. The amount of time that the Deering drainage basins
have been dried is unknown, however the desiccation of the depressions
in the Cutler Slough allowed for encroachment of vegetation to occur.The
unsystematic input of freshwater from the C-100A via pump station is
believed to have created stress on the encroached and surrounding
vegetation which may be causing individual organisms to alter their
physiology and morphology in response to changes in environmental
conditions'' (Schlichting, 1986). In conjunction with canal-derived
flow, the shape of the Deering drainage basin creates high-flow and
low-flow water regimes, where water may pool or completely dry due to
variability in ground elevation, porosity of substrate, and overall
influx of water added to the system. In addition, within the two basins,
water and nutrient uptake, and water quality conditions might differ
from point source input to output as the Deering Estate acts as a sink.
In response to the effect that the rehydration has on the two
drainage-basins, observations on vegetation growth or die-off may
clarify relationships between water regime and vegetation response
within a restoration context. It also could uncover species-specific
responses to flooding and nutrient stresses that result in vegetation
plasticity, physiological drought, reduced growth, and decreased
survival. The primary goal is to determine the effects of the CUtler
SLough's sporatic dehydration on tree species, and determine water and
nutrient uptake of tree species with varying health conditions, and
water quality within the Deering Estate's natural areas hardwood hammock
wetland habitats through three belt transects that cross perpendicularly
through the two basins: high-flow and low-flow.

\begin{center}\rule{0.5\linewidth}{\linethickness}\end{center}

\hypertarget{methods-dataset-and-statistical-analysis}{%
\section{Methods (Dataset and Statistical
Analysis)}\label{methods-dataset-and-statistical-analysis}}

\begin{itemize}
\tightlist
\item
  Identify physiognomic and physiological characteristics affecting tree
  species within the two drainage basins in their capacity to adapt to
  the present hydrology.\\
\item
  Evaluate nutrient uptake via plant nutrient analysis by various tree
  species within areas along the CUtler Creek recieving high flow, and
  areas in the hardwood hammocks wetland recieving low flow.\\
\item
  Measure water uptake via sap flow sensor on select tree species
  displaying both healthy and unhealthy physiological characteristics.\\
\item
  Compare resurvey of species-specific mortality data on woody
  vegetation with data from 2016.
\end{itemize}

\end{document}
